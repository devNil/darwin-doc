\documentclass{scrartcl}

\usepackage[ngerman]{babel}
\usepackage[utf8]{inputenc}
\usepackage{tabularx}

\usepackage[T1]{fontenc}
\newcommand{\changefont}[3]{
\fontfamily{#1} \fontseries{#2} \fontshape{#3} \selectfont}
\changefont{phv}{m}{n}

\begin{document}
	\section*{Arbeitsjournal Yannik Dällenbach}

	Datum : 26.04.2013
	\\

	\begin{tabularx}{\textwidth}{| m{2cm} | m{2cm} | X | X |}
	\hline
	Tatsächlicher Zeitbedarf & Geplanter Zeitbedarf & Beschreibung der Arbeit & Bemerkungen \\
	\hline
	1 Tag & 1 Tag & Implementieren des Tron-Klons & Das Protokoll musste erweitert werden. \\
	\hline
	\end{tabularx}

	\subsection*{Reflexion des Arbeitstages}
	Heute begann ich mit der Implementierung des Tron-Klons. Das Spiel sollte in 
	HTML5 realisiert werden. Zur grafischen Ausgabe benutze ich 
	das Canvas-Element. Insgesamt besteht das Frontend zum grössten Teil
	aus Javascript. Beim Implementieren des Javascript-Teils stellte sich heraus
	,dass das Protokoll welches wir in JSON realisiert haben noch nicht alle
	benötigten Funktionalitäten für das Spiel besitzt. Es fehlten zum Beispiel
	das Markieren von Spielern welche verloren haben. Deshalb musste ich das 
	Protokoll zuerst Serverseitig anpassen was keine riesige Aufgabe darstellte.
	Am Ende des Tages funktionierte das Spiel komplett und man konnte bereits die
	ersten Runden spielen.
	\subsection*{Pendenzen für den nächsten Arbeitstag}
	Die Implementierung des mobilen Controllers.
\end{document}
