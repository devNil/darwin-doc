\section{Systemdokumentation}
\subsection{Konfigurations-Dokumentation}
Um DARWIN verwenden zu können werden folgende Komponenten benötigt.
\subsubsection{git}
\textbf{git} wird als Versionierungssystem verwendet und um den 
Code von github zu clonen. 
\subsubsection{Go-Compiler Version 1.1}
Mithilfe des Go-Compilers wird der Sourcecode zu einem ausführbaren Binary
kompiliert. Im Allgemeinen wird die Go-Compiler Toolchain zum Auflösen 
der Dependencies von DARWIN und Kompilieren verwendet.
\subsubsection{Umgebungsvariablen}
Um DARWIN verwenden zu können muss das Setup Bash Script(setup.sh) ausgeführt werden.
Dies setzt alle benötigten Umbgebungsvariablen.
\begin{itemize}
    \item PORT
    \item TEMPLATE
\end{itemize}
\textbf{PORT}
\\
Die PORT-Umgebungsvariable spezifiziert an welchem  Port der Webserver sich
bindet.
\\
\textbf{TEMPLATE}
\\
Die TEMPLATE-Umgebungsvariable spezifiziert wo sich die HTML-Templates befinden. 
\subsection{Benutzerhandbuch}
Um das Proof of Concept zu testen sind folgende Punkte zu befolgen:\\
\begin{itemize}
    \item Installieren Sie die Go-Runtime (\url{http://golang.org/})
    \item Downloaden Sie das Proof of Concept (\url{https://github.com/devNil/darwin}) 
    \item Starten Sie das Proof of Concept mit dem Befehl \textbf{go run main.go}
\end{itemize}

\subsection{Supporthandbuch}
Für die technisch versierten Benutzer besteht die Möglichkeit Parameter des Proof of Concet zu ändern, diese befinden sich in der Datei game.go. \\
