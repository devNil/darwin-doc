\section{Ausgangslage}
\subsection{Problemstellung}
Moderne Videospiele enthalten meistens eine ausführliche Künstliche Intelligenz\footnote{Künstliche Intelligenz} welche sich jedoch nicht immer dem Spieler anpasst. Meist werden daher verschiedene Schwierigkeitsgrade angeboten, beispielsweise Leicht, Mittel und Schwer.\\
Genau hier setzt DARWIN an. Ziel unseres Projekts ist daher eine Künstliche Intelligenz zu entwickeln welche sich jedem Spieler indivuell anpassen kann und somit nach einigen Spielen zu einem fordernden Gegner mutiert. Die Entwicklung des Gegners wird mithilfe eines Evolutionären Algorithmus gewährleistet, die Implementierung selbst ist als Tron-Klon realisiert.
\subsection{Anlass und Begründung des Projekts}
DARWIN ist als lernfähige Künstliche Intelligenz konzipiert. Dies bedeutet das der Gegner anfangs extrem schlecht auf den Gegner eingestellt ist, später jedoch anspruchsvoller wird. Für Spieler entsteht eine spezielle Spielerfahrung, da die Künstliche Intelligenz immmer besser agiert. \\\\Für uns als Lehrnende eröffnet DARWIN ausserdem einen Einblick in das Thema Evolutionäre Algorithmen.
\subsection{Projektrahmenbedingungen}
\begin{itemize}
	\item Das Projekt DARWIN wird am 4.6.2013 22:30 abgeschlossen.
	\item Das Projekt DARWIN wird nach der Projektmethodik HERMES geführt.
\end{itemize}
\subsection{Situationsanalyse}
\subsection{Erbrachte Vorleistungen}
