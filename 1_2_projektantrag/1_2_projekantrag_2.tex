\section{Ausgangslage}
\subsection{Problemstellung}
Moderne Videospiele enthalten meistens eine ausführliche Künstliche Intelligenz
\footnote{Als Künstliche Intelligenz wird der Versuch bezeichnet eine menschenähnliche \"Intelligenz\" für das Problemverhalten nachzubilden.} welche sich jedoch nicht immer dem Spieler anpasst. Meist werden daher verschiedene Schwierigkeitsgrade angeboten, beispielsweise Leicht, Mittel und Schwer.\\
Genau hier setzt DARWIN an. Ziel unseres Projekts ist daher eine Künstliche Intelligenz zu entwickeln welche sich jedem Spieler indivduell anpassen kann und somit nach einigen Spielen zu einem fordernden Gegner mutiert. Die Entwicklung des Gegners wird mithilfe eines Evolutionären Algorithmus's gewährleistet. Die Implementierung selbst ist als Tron-Klon realisiert.
\subsection{Anlass und Begründung des Projekts}
DARWIN ist als lernfähige Künstliche Intelligenz konzipiert. Dies bedeutet das der Gegner anfangs extrem schlecht auf den Spieler eingestellt ist, später jedoch anspruchsvoller wird. Für Spieler entsteht eine spezielle Spielerfahrung, da die Künstliche Intelligenz immmer besser agiert und sich anpasst. \\\\Für uns als Lernende eröffnet DARWIN ausserdem einen Einblick in das Thema Evolutionäre Algorithmen. Die Realisierung dient als Proof of Concept.
\subsection{Projektrahmenbedingungen}
\begin{itemize}
	\item Das Projekt DARWIN wird am 4.6.2013 abgeschlossen.
	\item Das Projekt DARWIN wird nach der Projektmethodik HERMES geführt.
\end{itemize}
\subsection{Situationsanalyse}
\begin{itemize}
	\item Durch ein Projekt\footnote{\url{http://github.com/druic/moveo}} im letzten Semester besitzen wir schon Erfahrung in
	OpenGL\footnote{OpenGL stellt eine Schnittstelle zur Grafikkarte dar um 2D- und 3D-Szenen direkt auf der Grafikkarte darzustellen.}.
	\item Durch Revolt\footnote{\url{http://github.com/devNil/revolt}} und weiteren kleineren Spielprojekten konnten wir schon Erfahrungen
	tätigen im Bereich der Spieleentwicklung.
\end{itemize}
\subsection{Erbrachte Vorleistungen}
Bereits wurden kleinere Evolutionäre Algorithmen von uns implementiert, meist um mathematische Probleme zu lösen.
\\\\ Als Beispiel: \url{http://github.com/Np2x/evo}