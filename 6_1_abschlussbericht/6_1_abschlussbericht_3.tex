\section{Erfahrungen}
In diesem Projekt haben wir neue Technologien sowie Projektplanungsmethoden kennen gelernt. Um diese nicht zu vermischen werden wir unsere Erfahrungen zwischen diesen Punkten unterscheiden: 
\subsection{Technologie}
Anfangs wollten wir unser Projekt in Objective-C realisieren. 
Die ganze Architektur war ursprünglich anders geplant, so setzten wir unseren Fokus auf die Implementierung von Evolutionären Algorithmen. 
Als wir während der Realisierungsphase feststellen mussten das zum Einen Objective-C als Programmiersprache weniger geeignet ist 
für eine performante Implementierung eines Evolutionären Algorithmus und zum Anderen ein Evolutionärer Algorithmus bei 
einem Spiel wie Tron unsinnig ist, entschieden wir uns für einen allgemeinen Fokuswechsel.
\\
Wir änderten die Architekur zu einem bewährten Client-Server-Modell. Das Front-End sollte mithife von HTML5 realisiert werden. Im Back-End
wollten wir Go verwenden, da wir die "Last" eines Spiels optimal und performant verteilen wollten. Glücklicherweise hatten wir einen solche Fokuswechsel in
der Konzeptionsphase vermerkt. Somit endete das Projekt als ein Multiplatform-Multiplayer-Spiel.

\subsection{Projektplanungsmethode}
Leider stellte sich schon früh heraus das Hermes als gewählte Projektmethode für den Geschmack eines "Forschungsprojekts" unserer Meinung nach 
weniger geeignet ist. Ein grosses Problem von Hermes ist die Iterationsarmut, wir könnten in der Realität nie so schnell auf ein architekturielles Problem wie wir es
erfuhren reagieren. Hermes wirkt in diesem Bereich viel zu statisch im Vergleich mit Scrum. Ein weiteres Problem, allgemeines Problem ist das viele Dokumentieren und das
ewige Duplizieren von zuvor geschriebenen Phasen.
\\
Wir denken das Projektplanungsmethoden eine Daseinsberechtigung besitzen, jedoch Hermes nicht dem Optimum(vorallem in der Softwareentwicklung) entspricht.
\\
Fokus eines Projekts sollte das Produkt und die Inline-Dokumentation sein. 
