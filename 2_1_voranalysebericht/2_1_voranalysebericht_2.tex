\section{Ist-Aufnahme und Ist-Analyse}
\subsection{Beschreibung des Ist-Zustands}
Momentan besitzen Videospiele meist mehrere Schwierigkeitsgrade(\emph{Leicht}, \emph{Mittel} und \emph{Schwer}). Die Gegner passen sich meist
nicht am Spielestil des aktiven Spielers an. Dies drückt jedoch meist auf den Spielspass, da die virtuellen Gegner nicht in der 
Lage sind, sich auf einen bestimmten Spielestil einzustellen. Es enstehen meist Strategien welche immer den Sieg versprechen. Das Spiel
verliert dadurch automatisch an Wiederspielwert.
\subsection{Schwachstellenanalyse}
\subsubsection{Schwachstellen}
\textbf{Kein Anpassen an Spielstil}\\
Die Gegner reagieren meist unflexibel. Der Spieler kann dadurch unter- oder überfordert sein. Meist werden drei Schwierigkeitsgrade
angeboten, wobei die Balance zwischen diesen vielfach stark schwankt, somit ist es möglich das der Schwierigkeitsgrad \emph{Leicht} sehr einfach ist und \emph{Mittel} schon extrem schwierig sein kann. Spieler finden dadurch meist kein Schwierigkeitsgrad welcher zu ihnen passt.\\\\
\textbf{Kaum Wiederspielwert}\\
Falls sich eine bestimmte Strategie in einem angebotenen Schwierigkeitsgrad etabliert hat ist meist der Sieg garantiert.
Erarbeitete oder kopierte Strategien lassen sich vielfach ohne grösseren Aufwand auf einem anderen Schwierigkeitsgrad benützen. Es resultiert
ein Verlust des Wiederspielwerts, da der Spieler nach einiger Zeit immer gewinnen kann.
\subsubsection{Mögliche Ursachen}
Die Hauptursache der Schwachstellen ist eine undynamische und unflexible Künstliche Intelligenz. Dem Spieler fällt nach einigen Spielrunden gewisse Reaktionsmuster auf und kann dadurch seine Strategie anpassen. Dem Gegner ist dies meist nicht gewährt. Das Spiel wird zur Repetition.
\subsubsection{Mögliche Lösungen und Verbesserungen}
\textbf{Evolutionäre Algorithmen}
\\
Mithilfe Evolutionären Algorithmen kann sich der Gegner während mehreren Spielerunden hinweg entwickeln. Der Gegner kann somit Runde für Runde anspruchsvoller und individueller werden.
\\\\
\textbf{Künstliche Neuronale Netze}\footnote{Mithilfe von Künstlichen Neuronalen Netzen wird versucht das Gehirn des Menschen und der dadurch verbundene Lernprozess zu simulieren.}
\\
Durch die Verwendung von Künstlichen Neuronalen Netzen könnte eine Künstliche Intelligenz simuliert werden welche lernt auf die Strategie des
Spielers einzugehen und sich anspruchsvoller verhält.
\subsection{Zukünftige Entwicklung}
Durch das Einsetzen einer Lösung welche befähigt ist zu lernen fallen automatisch die Schwierigkeitsgrade weg. Spiele welche eine flexible Künstliche Intelligenz implementieren verlieren nicht an Wiederspielwert. Der Gegner wird in jedem Spiel anders auf die Strategien des Spielers reagieren und ihm ein immer anspruchsvolleres Spielerlebnis bieten. Lösungen können auch für das optimieren anderer Probleme benützt werden.