\section{Lösungsvarianten}
Folgende Lösungsvarianten entsprechen beide dem Ziel einer anpassbaren Künstlichen Intelligenz. Diese werden in folgenden Abschnitten genauer
erklärt. Es werden bewusst keine weitern Lösungsvarianten aufgeführt, da beide ausreichend sind.
\subsection{Lösung 1: Evolutionärer Algorithmus}
\subsubsection{Beschreibung der Lösungsvariante}
Die Lösung wird direkt vom Spieler trainiert. Dabei entwickelt sich die Künstliche Intelligenz durch mehrfaches Spielen. Sie wird dabei nur entfernt trainiert. Das Anpassen erfolgt durch durch die Entwicklung über mehrere Generationen. Die Künstliche Intelligenz benötigt kaum Problemwissen. Somit kann diese Lösung als portabel angesehen werden, das gleiche ausgearbeitete Verfahren könnte auch in anderen Spielen angwendet werden.
\subsubsection{Realisierbarkeitsbetrachtung}
Die Methodik hinter Evolutionären Algorithmen ist leicht erlernbar. Die einzige Schwierigkeit ist das Genom\footnote{Die Genetischen Informationen eines einzelnen Individuums.}, welches zuerst ausgearbeitet werden muss.
\subsection{Lösung 2: Künstliches Neuronales Netz}
\subsubsection{Beschreibung der Lösungsvariante}
Durch ein Künstliches Neuronales Netz kann die Künstliche Intelligenz trainiert werden und dadurch flexibel auf gewisse Inputs reagieren. Die Künstliche Intelligenz kann sich somit auch auf den Spielstil des Spielers anpassen.
\subsubsection{Realisierbarkeitsbetrachtung}
Die Realisierung eines Künstlichen Neuronalen Netzes ist schwierig. Das entstandene Künstliche Neuronale Netz muss weiterführend extern trainiert werden, um gewisse Grundfertigkeiten zu erlernen. Im Allgemeinen ist diese Lösung komplexer und dadruch anspruchsvoller. Sie erfordert ungemein viel Vorwissen.