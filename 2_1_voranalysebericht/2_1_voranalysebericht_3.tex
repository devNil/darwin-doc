\section{Systemziele}

Aus den vorderen Überlegungen erschliesst es sich schnell das unser primär Ziel darin besteht in einem Prof-Of-Concept aufzuzeigen wie sich Gegener durch lernfähiges Verhalten dem Spieler anpassen können, um diesen optimal zu fordern und zum regelmässigen spielen zu motivieren.  
\\ 
Unsere Prioritäten gehen von 1 bis 3, wobei 1 als niedrigste und 3 als höchste Priorität angeschaut wird.
\\\\
\begin{tabularx}{\textwidth}{| p{0.7cm} | X | p{1.5cm} | X |}
\hline
\rowcolor[gray]{0.9} ID & Beschreibung & Priorität & Messkriterium\\
\hline
Z1 & Das Prof-Of-Concept ermöglicht es zwei Spielern gegeneinander anzutreten. & 3 & Sobald ein Tron-Klon spielfähig ist und über die wichtigsten Funktionen des Spieles verfügt.\\
\hline
Z2.1 & Dem Spieler steht zusätzlich die möglichkeit zur verfügung gegen Computer-Gegener zuspielen. & 3 & Gegner verfügen über eine simple AI welche ihnen die Basisfunktionalität im Spiel ermöglicht. \\
\hline
Z2.2 & Gegner verfügen über ein lernfähiges Verhalten und passen sich dem Spiel an. & 3 & Dieses Ziel ist daran zumessen ob ein Gegner sich dem Spieler anpasst. \\
\hline 
Z2.3 & Gegner können sich selber trainieren ohne dabei auf Menschen angewiesen zu sein.  & 2 & Den Gegnern wird eine Arena geboten wo sie sich selber trainieren. \\
\hline
Z3 & Der Spieler kann verschiedene Gegner gegen sich spielen lassen um sie zu trainieren. & 2 & Der Spieler hat die Möglichkeit aus verschiedenen Gegner einzelne zu trainieren. \\
\hline
Z4 & Spieler lassen ihre Gegner gegeneinander Antretten & 1 & Dem Spieler wird die Möglichkeit geboten seine Gegner, nach dem Trainieren, gegen andere Spieler oder deren Gegner zu spielen. \\
\hline
\end{tabularx}