\documentclass{scrartcl}

\usepackage{tabularx} %better tables
\usepackage[table]{xcolor}
\usepackage[utf8]{inputenc}
\usepackage[ngerman]{babel}
\usepackage{hyperref}
\usepackage{tikz}

\begin{document}
	\section*{Projektantrag}
	
	\begin{tabularx}{\textwidth}{| X | X |}
	\hline
	Status & In Arbeit\\
	\hline
	Projektname & DARWIN\\
	\hline
	Projektleiter & Noe Thalheim\\
	\hline
	Auftraggeber & Stefan Schenk\\
	\hline
	Autoren & Yannik Dällenbach, Noe Thalheim\\
	\hline
	\end{tabularx}
	
	\subsection*{Änderungskontrolle}
	\begin{tabularx}{\textwidth}{| X | X | X |}
	\hline
	\rowcolor[gray]{0.9} Version & Datum & Beschreibung\\
	\hline
	1.0 & 19.02.2013 & Init\\
	\hline
	\end{tabularx}

	\subsection*{Referenzen}
	
	\begin{tabular}{| l | l | }
	\hline
	\rowcolor[gray]{0.9} Referenz & Quelle, Datum\\
	\hline
	Xcode & https://developer.apple.com/technologies/tools/, 19.02.2013\\
	\hline
	moveo & http://github.com/druic/moveo, 19.02.2013\\
	\hline
	revolt & http://github.com/devNil/revolt, 19.02.2013\\
	\hline
	evo & http://github.com/Np2x/evo, 19.02.2013\\
	\hline
	Github & http://github.com/, 19.02.2013\\
	\hline
	 & \\
	\hline
	\end{tabular}
	
	\pagebreak
	%inhaltsverzeichnis
	\tableofcontents
	\pagebreak
	%abschnitt 1
	\section{Zweck des Dokuments}
Dieses Dokument enthält den Projektantrag für das Projekt DARWIN. Es beschreibt kurz die Ausgangslage, Ziele und Lösungen, sowie voraussichtlich benötigte Mittel für die Durchführung des Projektes. Es gibt im Weiteren Auskunft über die Konsequenzen und Risiken, welche mit dem Projekt verbunden sind.
	%abschnitt 2
	\section{Ausgangslage}
\subsection{Problemstellung}
Moderne Videospiele enthalten meistens eine ausführliche Künstliche Intelligenz\footnote{Künstliche Intelligenz} welche sich jedoch nicht immer dem Spieler anpasst. Meist werden daher verschiedene Schwierigkeitsgrade angeboten, beispielsweise Leicht, Mittel und Schwer.\\
Genau hier setzt DARWIN an. Ziel unseres Projekts ist daher eine Künstliche Intelligenz zu entwickeln welche sich jedem Spieler indivuell anpassen kann und somit nach einigen Spielen zu einem fordernden Gegner mutiert. Die Entwicklung des Gegners wird mithilfe eines Evolutionären Algorithmus gewährleistet, die Implementierung selbst ist als Tron-Klon realisiert.
\subsection{Anlass und Begründung des Projekts}
DARWIN ist als lernfähige Künstliche Intelligenz konzipiert. Dies bedeutet das der Gegner anfangs extrem schlecht auf den Gegner eingestellt ist, später jedoch anspruchsvoller wird. Für Spieler entsteht eine spezielle Spielerfahrung, da die Künstliche Intelligenz immmer besser agiert. \\\\Für uns als Lehrnende eröffnet DARWIN ausserdem einen Einblick in das Thema Evolutionäre Algorithmen.
\subsection{Projektrahmenbedingungen}
\begin{itemize}
	\item Das Projekt DARWIN wird am 4.6.2013 22:30 abgeschlossen.
	\item Das Projekt DARWIN wird nach der Projektmethodik HERMES geführt.
\end{itemize}
\subsection{Situationsanalyse}
\subsection{Erbrachte Vorleistungen}

	%abschnitt 3
	\section{Ziele und Lösungen}
\subsection{Zielvorstellungen}
\textbf{Bezogen auf die Künstliche Intelligenz}
\begin{itemize}
	\item Ein Evolutionärer Algorithmus ist ausgearbeitet für das Verhalten der Künstlichen Intelligenz.
	\item Die Künstliche Intelligenz passt sich nach einigen Spielen dem Spieler an.
\end{itemize}
\textbf{Bezogen auf den Tron-Klon}
\begin{itemize}
	\item Das Spiel ist spielbar.
\end{itemize}
\subsection{Mögliche Lösung}
\textbf{Bezogen auf die Künstliche Intelligenz}
\begin{itemize}
	\item Ein Evolutionärer Algorithums wird verwendet um sich dem Spielverhalten des Spielers anzupassen.
	\item Drei Schwierigkeitsgrade existieren (Leicht, Mittel und Schwer).
\end{itemize}
\textbf{Bezogen auf den Tron-Klon}
\begin{itemize}
	\item Das Spiel wird mit der Standard-Grafik-API\footnote{Application programming interface, eine Programmierschnittstelle.} dargestellt.
	\item Das Spiel wird mithilfe von OpenGL dargestellt.
\end{itemize}
		\section{Mittelbedarf}
	
	\subsection{Sachmittel}
	
	Um unser Projket umzusetzen können wir auf unsere bestehende Hardware zurückgreiffen, als Software werden wir Xcode verwenden welches ebenso vorhanden ist. 
	Da Xcode über eine native Git integration verfügt werden wir als Versionierungssystem Git verwenden und das Repositorie auf Github hosten. 
	\\ \\
	Für die Informationsbeschaffung ist es nicht auszuschliessen das wir uns im Verlauf des Projkets Bücher kaufen müssen. 
	
	\subsection{Personal / Ausbildung}
	
	Unser Projekt befasst sich mit diversen Themengebieten bei denen wir entweder keine oder nur sehr wenig Erfahrung mitbringen, diese Gebiete werden wir in folge dieses Projekts kennenlernen und uns intensiv damit beschäfftigen. 
	
	\subsection{Dienstleistung}
	Zum jetzigen Zeitpunkt werden wir bei den Dienstleistungen wahrscheinlich nur auf Git und Github zugreiffen.
	\usetikzlibrary{arrows,positioning}
	
	\section{Planung und Organisation}
	
	\subsection{Projektorganisation}
	An dem Projekt werden folgende Personen mitarbeiten:
	\\\\
	\begin{tikzpicture}[node distance=1cm, auto]  
	\tikzset{
	    mynode/.style={rectangle,align=left,draw=black, top color=white, bottom color=white!50,very thick, inner sep=1em, minimum size=3em, text centered},
	    myarrow/.style={->, >=latex', shorten >=1pt, thick},
	    mylabel/.style={text width=7em, text centered} 
	}  
	\node[mynode] (auftraggeber) {\textbf{Auftraggeber}\\Stefan Schenk};  
	\node[mynode, below=1cm of auftraggeber] (projektleiter) {\textbf{Projektleiter}\\Noe Thalheim};  
	\node[mynode, right=of projektleiter] (projektmitarbeiter1) {\textbf{Projektmitarbeiter}\\Yannik Dällenbach};


	\draw[myarrow] [<->] (auftraggeber.south) -- (projektleiter.north);	
	\draw[myarrow] [-] (projektleiter.east) -- (projektmitarbeiter1.west);

	\end{tikzpicture} 
	\medskip

	Die Aufgabenverteilung wird das ganze Projekt lang gleich bleiben.
	
	\subsection{Termine}
	
	Für unser Projekt sind folgende Termine von Wichtigkeit:
	\begin{description}
		\item[] Starttermin: 05.02.2013
		\item[] Endtermin: 04.06.2013
	\end{description}
	Daraus ergibt sich für unser Projekt folgender Terminplan:
	\\ \\
	\tiny{
	\begin{tabular}{| p{2cm} | p{1cm} | p{1cm} | p{1cm} | p{1cm} | p{1cm} | p{1cm} |}
	\hline
	\rowcolor[gray]{0.9}  & 12.02.13 - 19.02.13 & 19.02.13 - 05.03.13 & 5.03.13 - 19.03.13 & 19.03.13 - 07.05.13 & 7.05.13 - 21.05.13 & 21.05.13 - 04.06.13 \\
	\hline
	Initialisierung & \cellcolor{yellow} & & & & & \\
	\hline
	Voranalyse & &  \cellcolor{yellow}& & & & \\
	\hline
	Konzept & & &  \cellcolor{yellow}& & & \\
	\hline
	Realisierung & & & &  \cellcolor{yellow}& & \\
	\hline
	Einführung & & & & & \cellcolor{yellow}& \\
	\hline
	Abschluss & & & & &  &\cellcolor{yellow} \\
	\hline
	\end{tabular}	
	}
	\small{
	\subsection{Prioritäten}
	Unsere Prioritäten für dieses Projekt sind das Verstehen und Implementieren von Evolutionären Algorithmen und somit das Selbststudium in diesen Gebieten. Zudem ist es uns wichtig durch dieses Projekt Erfahrungen zusammeln im Zusammenhang mit der Programmierung in Objective-C. 
}
	\documentclass{scrartcl}

\usepackage{tabularx} %better tables
\usepackage[table]{xcolor}
\usepackage[utf8]{inputenc}
\usepackage[ngerman]{babel}

\begin{document}
	
	\section{Wirtschaftlichkeit}
	Das Projekt DARWIN soll in einem Prof-Of-Concept aufzeigen wie sich Evolutionäre Algorithmen bewähren, Gegner in einem Spiel dem Spieler anzupassen. Das Projekt hat keinen Wirtschaftlichenaspekt im allgemeinen Sinne, könnte jedoch als Grundlage für ein späteres Projekt dienen, welches sich zu einem Spiel entwicklen könnte.
	
\end{document}
	\documentclass{scrartcl}

\usepackage{tabularx} %better tables
\usepackage[table]{xcolor}
\usepackage[utf8]{inputenc}
\usepackage[ngerman]{babel}

\begin{document}
	
	\section{Konsequenzen und Risiken}
	
	\subsection{Risikobeurteilung}
	Da es sich bei unserem Projekt um Themen handelt bei denen keines der Projektmitglieder auf grosse Erfahrung bauen kann ist das grösste Risiko, das wir Probleme mit der Realisierung der Evolutionären Algorithmen haben werden und somit unser Prof-Of-Concept nicht fertig stellen könnten. 
	Den Tron-Clon auf welchem unser Prof-Of-Concept basiert können wir hingegen ohne grossen Aufwand realisieren.
	
	\subsection{Ausweichmöglichkeiten}
	Falls es tatsächlich soweit kommen sollte das es uns nicht möglich ist die von uns erwähnten Evolutionären Algorithmen zu implementieren werden wir uns Hilfe von aussenstehenden Organisieren oder die Evolutionären Algorithmen weglassen und uns den Tron-Clon konzentrieren. 
\end{document}
	\documentclass{scrartcl}

\usepackage{tabularx} %better tables
\usepackage[table]{xcolor}
\usepackage[utf8]{inputenc}
\usepackage[ngerman]{babel}

\begin{document}
	
	\section{Antrag}
	Wir beantragen die Genehmigung des vorliegenden Projektantrages und die Freigabe für die Phase Voranalyse durch den Auftraggeber.
	\\ \\ \\
	Noe Thaleim:
	\\ \\
	\parbox{4cm}{\hrule
	\strut \centering\footnotesize Ort, Datum} \hfill\parbox{4cm}{\hrule
	\strut \centering\footnotesize Unterschrifft}
	\\ \\ \\
	Yannik Dällenbach:
	\\ \\
	\parbox{4cm}{\hrule
	\strut \centering\footnotesize Ort, Datum} \hfill\parbox{4cm}{\hrule
	\strut \centering\footnotesize Unterschrifft}
	\\ \\ \\
	Für den Auftraggeber:
	\\ \\
	\parbox{4cm}{\hrule
	\strut \centering\footnotesize Ort, Datum} \hfill\parbox{4cm}{\hrule
	\strut \centering\footnotesize Unterschrifft}
	
\end{document}
	
\end{document}